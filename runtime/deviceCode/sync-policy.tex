MERCATOR's runtime involves complex interactions among many classes,
most of which include some form of global memory state variables.  In
addition, some local variables used to communicate between threads in
different warps may be stored in per-SM shared memory.  When different
threads of a block read and write this data, there is the potential
for unpredictable read/write ordering hazards.  Even within a single
warp, newer NVIDIA GPU architectures give each thread a separate
program counter, so it is NOT safe to assume that threads in one warp
execute in lockstep.  Hence, no matter how many warps constitute a
block, it is necessary to use synchronization to protect interthread
communication through global memory.

MERCATOR's runtime uses the following policies to ensure safe
synchronization across threads:

(1) The 0th thread of a block is the ``boss'' thread.  Certain
variables and data structures are only read and written by the boss;
these structures do not require synchronization to protect them.
Other variables and structures are written by the boss but potentially
read by any thread in the block.  The latter data must be protected
by appropriate __syncthreads() or similar calls.

There are relatively few examples of threads other than the boss
writing shared or global data in the runtime.  The obvious cases are
writes of data to the inter-node queues and the input and output
stream buffers. Queue writes are guaranteed safe from subsequent reads
so long as there is a sync between successive calls to node fire()
functions.  Writes to the output stream are safe because it is never
read by the kernel.

User nodes may write their own state using multiple threads, but this
data is disjoint from the runtime's own data structures, and the user
is responsible for managing it appropriately.  Nodes' writes to queues
all occur via the runtime's push(), which appropriately manages access
to these queues.

(2) Every function in the runtime is one of four types with respect
    to global and shared state:

  (0) functions called only from the init or cleanup kernels, not
      the main kernel.  These functions are all run with one thread
      for each block, and each block's thread touches non-overlapping
      data structures, so synchronization in tehse functions is not
      necessary.

  (i) read-only [generally const]

  (ii) read-write, boss only (marked with 'assert(IS_BOSS())' at the
       head of the function.

  (iii) read-write, called with all threads.  These functions are
      responsible for protecting their own writes of global and shared
      data from reads by ANY function in the runtime.

There are currently almost no instances in the runtime of data-writing
functions that run with some but not all threads.  The exceptions are
dsWrite() for channels, which touches only the memory of the
downstream queues (which will be sync'd in scheduler before it is read
by another node) and push() for buffered nodes, which is carefully
designed so that each thread touches non-overlapping storage within
one call, and the subsequent finishWrite() function ensures that
synchronization happens before the written data is read again.

There are a number of cases in which a node needs to do some operation
for each of its output channels.  While some of these operations could
in principle be parallelized across threads, doing so would result in
multiple threads potentially writing data structures that are
currently written only by the boss and so would require careful
analysis to ensure safety.  (Moreover, different channels are
instantiations of a template class with different data types, so the
parallel invocations would be running divergent code.)

(3) When a multithreaded function writes global or shared state, the
function MUST ensure that any such writes are safe with respect to
code outside this function.  To ensure safety, the function must
synchronize both BEFORE the write and AFTER the write.  Syncing before
writing a variable ensures that threads relying on the pre-write value
will complete before the value changes.  Syncing after the write
ensures that all threads that expect the post-write value will see it.

Every sync is annotated with a comment indicating the variables and
data structures for which it is a pre-write or a post-write sync. Note
that calls to boss-only, non-read-only functions may contain writes
that must be protected by the caller, because all threads, not just
the boss, must participate in a sync.

NOTES ON EFFICIENT SYNCING BEHAVIOR

We assume WLOG that there is always a sync between successive calls to
nodes' fire() functions.  This sync is currently needed in the
scheduler to protect the broadcast of the next fireable node to all
threads, but even if it were not needed, the sync should be maintained
so that we don't have to protect every write to the data queues.
There may be some other writes that no longer need protection within
the scope of a single fire() call.

The main need for syncs on data that cross function boundaries in the
runtime is to protect the tail pointers of the downstream
queues. These are used before the next downstream write to determine
if the downstream queues are full, and to get their occupancy for
enumerate and source nodes.  The enumerate node and signal handling
also incur syncs to protect writes to per-node state that can be
updated by signals.  Much of the other state needed in firing
(e.g. queue head ptrs) changes only at the end of the firing.

Many other syncs are incurred on __shared__ variables that are used
for broadcast.  In particular, the boss thread computes some value
that must immediately be communicated to all threads in the block and
uses the shared variable to communicate it.  If a block were to have
only a single warp, these shared variables could be replaced with
shuffle_sync operations, which would effectively implement the sync
BEFORE write and eliminate the need for a separate sync AFTER write.
It might be worth implementing a generic broadcast<T> operation to
encapsulate the implementation here.

Data structures that are ``boss-only'' in the runtime include:
 - the worklist and its top pointer in the scheduler
 - the entire state of the parent buffer (but *NOT* the parent index
   in each NodeFunction)
 - some scheduling state of each node (active, flushStatus), but *not*
   nDSActive or the blocking flag, which are read by all threads.
 - the numItemsWritten counter in channels, used to set credit for
   outgoing signals.
